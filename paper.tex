\documentclass[letterpaper, twocolumn]{article}
\usepackage{adfundem}
\usepackage{hyperref}
\usepackage{lipsum}
\pagenumbering{gobble}



\title{Ad Fund 'Em -- Enabling Advertising in \LaTeX \ to Aid Academic Funding in a Time of Austerity}

\author{
	K.W. van Hove \\ University of Twente\\
	k.w.vanhove@utwente.nl
}

\date{April 4, 2025}





\begin{document}
	\maketitle
	
	\begin{abstract}
		Funding in academia is increasingly at risk, requiring researchers and academics to come up with alternative funding sources. Outside of academia, advertising is a popular source of revenue for many publications such as magazines and newspapers.
		
		In this paper we show how advertising in academic publications can unlock an alternative source of revenue for academics, which, with government funding worldwide on the decline, might prove a fruitful way to keep existing research ventures alive.
		
		To that extend we create ``Ad Fund 'Em'', a \LaTeX \ package which automatically adds advertisements throughout a manuscript.
	\end{abstract}
	
	
	\section{Background}
	The financial stability of academic institutions has long been a subject of concern, but recent developments have exacerbated the precariousness of funding within higher education and research. A combination of declining public investment, shifting political priorities, and increasing reliance on competitive grants has placed significant strain on scholars and institutions alike. This paper explores the primary factors contributing to the current crisis in academic funding and the potential consequences for research and education.
	
	One of the primary threats to stable academic funding is the decline in public investment. Many countries have reduced direct government support for universities, shifting financial responsibility toward tuition fees, private donations, and external grants. In the United States, for example, state funding for public universities has steadily decreased over the past few decades, leading to increased tuition costs and greater financial burdens on students. Similarly, European universities have faced budget cuts, particularly in countries that have adopted austerity measures following economic downturns. The reduction in public funding limits the ability of institutions to invest in research infrastructure, faculty hiring, and student support services.
	
	The erosion of stable funding sources has far-reaching consequences for both research and education. Reduced funding limits universities' ability to support early-career researchers, leading to precarious employment conditions such as short-term contracts and adjunct positions. Inadequate funding also hampers the advancement of fundamental research, as scholars are compelled to prioritize projects with immediate practical applications over long-term scientific discovery. Moreover, the financial strain on universities often leads to increased student tuition fees, reducing access to higher education for lower-income populations and exacerbating socioeconomic inequalities.
	
	\section{First Level Heading}
	\lipsum[9]
	
	\subsection{Second Level Heading}
	\lipsum[10]
	
	\subsubsection{Third Level Heading}
	\lipsum[11]
	
	\paragraph{Fourth Level Heading}
	
	Fourth level headings must be flush left, initial caps and roman type.
	One line space before the fourth level heading and $1/2$ line
	space after the fourth level heading.
	
	\subsection{Citations In Text}
	
	Citations within the text should indicate the author's last name and
	year\cite{Knuth-vol3}. Reference style\cite{Comer-btree}
	should follow the style that you are used to using, as long as the
	citation style is consistent.
	
	\subsubsection{Footnotes}
	
	Indicate footnotes with a number\footnote{This is a sample footnote} in
	the text. Place the footnotes at the bottom of the page they appear on.
	Precede the footnote with a vertical rule of 2 inches (12 picas).
	
	\subsubsection{Figures}
	
	All artwork must be centered, neat, clean and legible. Do not use pencil
	or hand-drawn artwork. Figure number and caption always appear after the
	the figure. Place one line space before the figure, one line space
	before the figure caption and one line space after the figure caption.
	The figure caption is initial caps and each figure is numbered
	consecutively.
	
	Make sure that the figure caption does not get separated from the
	figure. Leave extra white space at the bottom of the page to avoid
	splitting the figure and figure caption.
	
	Figure \ref{fig1} shows how to include a figure as encapsulated postscript.
	The source of the figure is in file {\tt fig1.eps}.
	
	\begin{figure}[ht]
		\begin{center}

			\caption{Sample EPS figure }
			\label{fig1}
		\end{center}
	\end{figure}
	
	Below is another figure using LaTeX commands.
	
	
	\begin{figure}[ht]
		\begin{center}
			\setlength{\unitlength}{1pt}
			\footnotesize
			\begin{picture}(160,80)
				\put(0,0){\framebox(160,80)[]{}}
				\put(10,35){\framebox(80,40){}}
				\put(100,20){\framebox(40,20){}}
				\put(70,10){\framebox(20,10){}}
				\put(20,5){\framebox(10,5){}}
			\end{picture}
			\caption{Sample Figure Caption}
		\end{center}
	\end{figure}
	
	\subsubsection{Tables}
	
	All tables must be centered, neat, clean and legible. Do not use pencil
	or hand-drawn tables. Table number and title always appear before the
	table.
	
	One line space before the table title, one line space after the table
	title and one line space after the table. The table title must be
	initial caps and each table numbered consecutively.
	
	\begin{table}[ht]
		\begin{center}
			\caption{Sample Table}
			
			\bigskip
			
			\begin{tabular}{|l|l|r|}
				\hline
				A & B & 1\\ \hline
				C & D & 2\\
				E & F & 3\\ \hline
			\end{tabular}
		\end{center}
	\end{table}
	
	
	\subsubsection{Handling References}
	
	Use a first level heading for the references. References follow the
	acknowledgements.
	
	
	\subsubsection{Acknowledgements}
	
	We want to thank \href{https://www.freepik.com/author/graphicforest}{GraphicForest} for the excellent advertisement placeholders.
	
	
	
	
	%\bibliographystyle{alpha} 
	%\bibliography{samplebib}
	%inline the .bbl file directly for mailing to authors.
	
	\begin{thebibliography}{Com79}
		
		\bibitem[Com79]{Comer-btree}
		D.~Comer.
		\newblock The ubiquitous b-tree.
		\newblock {\em Computing Surveys}, 11(2):121--137, June 1979.
		
		\bibitem[Knu73]{Knuth-vol3}
		D.~E. Knuth.
		\newblock {\em The Art of Computer Programming -- Volume 3 / Sorting and
			Searching}.
		\newblock Addison-Wesley, 1973.
		
	\end{thebibliography}
	
\end{document}
